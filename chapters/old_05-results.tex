\chapter{Findings}

\section{How do developers learn about accessibility?}

\subsection{Relatable examples to gain interest}
Stian uses examples that everyone, whether you have a disability or not, can relate to when he teaches accessibility. These examples are to motivate accessibility and universal design is important. Examples he uses:
\begin{itemize}
    \item Sun glare on computer screens to demonstrate the issue of having enough contrast between elements.
    \item Wanting to watch a video without a headset to demonstrate the need for subtitles.
\end{itemize}

Stian has also used parkinsons gloves that has a spinning motor simulating parkinsons disease and Cambridge Glasses on a stand at internal seminars. The people who came to the stand were asked to use a smartphone with the Parkinson gloves and to read on a poster with text with varying contrast levels between the text and the background. Stian notes:

\begin{displayquote}
    With the glasses you can't see anything under the requirement level.
\end{displayquote}

Stian also uses examples because going straight into the requirements and laws would be detrimental.

\begin{displayquote}
    It is quickly more interesting for developers to follow examples rather than rules and requirements. It is ironically very difficult to learn about all the Universal Design requirements. It is hard to relate to them. I don’t think WCAG 2.0 is easy, but now I have used it so much, so I am used to it. When I started with it, it was hairy.
\end{displayquote}

Stian notes that actually experiencing what thinking about accessibility  (or the lack of it) can do is much more effective than reading about requirements.

\begin{displayquote}
    To see someone struggling with the amazing thing you think you have built helps and creates engagement - builds empathy. Instead of reading about it, experience it. It has a much bigger effect.
\end{displayquote}

\subsection{Prestigious}
Stian says that when he got interested in accessibility it was prestigious to make semantic and validated HTML code. Writing semantic and validated HTML code often solves many accessibility related problems. \textcite{sandnes} says: "Computer students are often fascinated by achieving near impossible things, being it novel and effective algorithms, making hardware do things it was not meant to, etc". Sandnes uses this as an approach to develop a positive attitude amongst computer science students toward universal design.

%The general impression I got from watching developers use the empathy tools, was that they had fun. They were joking and laughing and seemed to enjoy doing it. 

\section{How do developers learn accessibility?}

\subsection{Enthusiast driven}
Stian says that some developers learns about universal design and accessibility in ICT projects, but that it relies heavily on having an enthusiast on the team that knows about it and is willing to share his / her knowledge to the rest of the team. Lars has also seen the importance of enthusiasts in ICT projects 

\begin{displayquote}
    (...) we have seen that if you have one enthusiast that can make sure you always have some focus on it if you forget about it in a project or in a company.
\end{displayquote}

Lars mentioned an enthusiast who had struggled with convincing his management to focus on universal design from the start rather than retrofitting in the late stages of the projects. The enthusiast had gone so far as to organise three projects which were quite similar and he instructed one of them to focus on universal design from the start, and the others to not focus on it. The project which focused on it from the start used 15\% less money then the project that had to retrofit implementation of universal design. The person who initiated this experiment did it to get hard numbers as arguments on why they should focus more on Universal Design.

\subsection{Random entry}
Lars says that you have no relation to universal design or accessibility if you have not gotten a task that governs it. He says the way into the field seems random. *


\section{The use of empathy tools}
Using empathy tools conveyed frustration, relief and surprise. Karl was using Keyboard Kim to order a ticket on SAS, when he got cought in a different ticket then he was going for.

\begin{displayquote}
    Oh, how does this work? What I did was that I entered the meny and I noticed that the radio buttons that I was going to use was not that easy to select from.
    
    There. Oh, I have to use the arrow keys, then. Oheh, help. I was caught in a flexible ticket. Because when I'm activating this choice, my pointer disappears. What I think I do now is that I go from 08:20 in the morning. I can't scroll by the way. Like that. Then I go back 19:35.
\end{displayquote}

Karl noticed many errors in the interface, and was not shy to express his frustration. When he had tried to buy a youth ticket for a while without succeeding, I said that we could move on.

\begin{displayquote}
    Yes, but I have to do those choices. Oh my god. It traps me all the time. It is just awful. There.
    
    Yes. Then... Ah, come on now. Then I go to order.
\end{displayquote}

It was clear that it was important for him to succeed at his task.

