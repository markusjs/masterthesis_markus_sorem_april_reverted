\chapter{Methods}

In this chapter I present the methods I have used to gather and analyse empirical data. I have only used qualitative data collection methods. This thesis has been exploratory, meaning that the goal was not to end up with a solution, but to explore the field of universal design of ICT solutions.

\rowcolors{2}{gray!30}{white}
\begin{table}[]
    \centering
    \caption{My caption}
    \label{my-label}
    \begin{tabular}{llr}
    \hline
    \rowcolor{gray!50!} \textbf{Participant(s) / setting} & \textbf{Method} & \textbf{Section}\\
    Animal    & Description                                        & Price (\$) \\ \hline
    Gnat      & per gram                                           & 13.65      \\
    Knot      & \begin{tabular}[c]{@{}l@{}}each\\ two\end{tabular} & 0.01       \\
    Gnu       & stuffed                                            & 92.50      \\
    Emu       & stuffed                                            & 33.33      \\
    Armadillo & frozen                                             & 8.99       \\ \hline
    \end{tabular}
\end{table}
%intervjuer
%eksperter
%
%\chapter{Methods}

\section{Research paradigm}
This thesis belongs in the critical research paradigm. According to \textcite{myers_set_2011} critical research in information systems is "...concerned with social issues such as freedom, power, social control". 

As the theme of my research question is to question why so few ICT solutions in Norway are meeting the requirements for Universal Design, and that the goal of Universal Design of ICT is to provide equal access to ICT services for most individuals of a society, the critical paradigm is a good fit for this project. 

%I am also interested in the target users view on atypical users and universal design, as well as to see if simulations can be a way of motivating the target users to create universal designed ICT solutions. // passer kanskje ikke her

Critical research aims to help realise human potential by "...eliminating causes of unwarranted alienation and domination" \textcite{myers_set_2011}. In this context, elimination of causes of unwarranted alienation and domination would be to figure out how to software simulation can help the target users in making ICT solutions which follows Universal Design guidelines. 

%"...the design of products and environments to be usable by all people, to the greatest extent possible, without the need for adaptation or specialised design" \parencite{miljoverndepartementet_t-1468_2007} this research is indeed critical.

%Placing the research in the interpretive area of research means that I am interested in how 



%I'm placing my research in the interpretative area of research, as I want to understand, not measure people.

\section{Focus Group}
Focus groups are simply a gathering of people who can discuss experiences and thoughts about specific topics with a researcher and each other \textcite[90]{CrangMike2007De}. These groups can sometimes have contradictory views, which can be a good thing. Contradictory and competing views can enable "spaces of resistance" where collective knowledge generation can happen \parencite[90]{CrangMike2007De}. 

To enable more participants to argue or discuss and reflect over a given topic can be valuable, both for the participants and for the researcher. The probability for whether these spaces of resistance can occur, relies on the group dynamic. It can be an advantage to recruit people who know each other if the goal is to study this group in particular. However, according to \textcite{CrangMike2007De}, recruiting strangers can facilitate the discussion to be more like the one they might have with strangers and encourage shy participants to be engaged in the conversation if less shy members "break the ice".

Whether the group is homogeneous or heterogeneous regarding gender can also impact the group dynamic and discussion in the focus group. Men have a tendency to "speak to the crowd" while women tend to break into one-on-one conversations which are lost to the group as a whole \parencite[92]{CrangMike2007De}.

\section{Interviews}
Interviews are usually categorised as unstructured, structured and semi-structured (Fontana and Frey 1994 in \cite{rogers_interaction_2011}). The choice of interview type depends on the research question and the goal of the interview.

\subsection{Expert interviews}
According to \citeauthor{meuser2009expert}, expert interviews can be an effective method of inquiry because an expert can be seen as "surrogates for a wider circle of players" \parencite[2]{meuser2009expert}. Especially in the exploratory phase of a project, expert interviews can be an efficient point of entry into the research area and can provide clues on where to go for further inquiry.  

An expert is someone who has has expert knowledge of a subject and is distinguished from someone with "everyday knowledge and common-sense knowledge" \parencite[18]{meuser2009expert}. Who is identified as an expert is up to the researcher's judgement \parencite{meuser2009expert}. 

Experts can inherit experiences, insider knowledge and domain knowledge about a subject. Experts can also be motivated to provide the researcher with knowledge to "make a difference" about the area of research or be motivated out of professional curiosity \parencite{meuser2009expert}.

\section{Casting the net}
%\section{Background activities}
Before conducting any empirical activities, I have investigated the topic through some background activities. I have observed a focus group held by students at my faculty, I went to an experience conference where the topic was Universal Design and I went to a meetup event held by an interest group called “Universal Design and digital inclusion”. Details on these activities are found in chapter xx.xx.

\subsection{Focus group with inexperienced developers}
The aim of the focus group was to look at the views and knowledge inexperienced developers had on Universal Design. The reason I went to this focus group was to get a better understanding of the views developers has on the subject of Universal Design and to get ideas on where I might steer the thesis.

Many of the participants had heard of Universal Design, but few knew exactly what it means or how to implement it. The developers did as they were told by their project leaders, and if they were requested to implement an accessibility fix, they did that. Some participants requested a better focus on Universal Design in higher education, as it would make their jobs easier if they had had more about the topic during their education.

\subsection{Experience conference}
I went to this conference to get a better understanding of Universal Design and how other fields (not related ICT) talk about this topic. I heard talks by architects, project managers and others who all had the same goal: to create a more inclusive society. 

Experiences I took from this conference:
\begin{itemize}
\item In all fields that can influence the experience people with disabilities might have with the world, there is a need to focus on Universal Design, especially in the education system.
\item On this conference they focused more on the inclusion of all members of society, but not on minimal requirements such as ramps and accessibility tools. 
\item The dogma on this conference seemed to be that facilitation for everyone, not just people with disabilities can benefit the society as a whole. In the opening statement, Helge Eide said: “Universal Design makes it possible for most people to participate in society as long as possible”. 
\item It was also mentioned that a visionary view on Universal Design can prohibit the evolution of the concept, and that we should focus on concrete examples. Universal Design must move from something you talk about in “party speeches” to something that happens in practise.
\item All Norwegian municipalities are obligated to have a council for people with impairments consisting of users with disabilities, politicians and representatives from the municipality. The council usually function as evaluators of solutions, but for this to work properly, this council should be included at the start of projects. 
\end{itemize}

The experience conference left me with the impression that there is a struggle to make the world accessible for all, but that this struggle will diminish if there is more focus on the topic in education programs for people who has a direct influence in artefacts and environments. It also left me with the impression that there often is a focus on minimal requirements in building projects so as not to break the law, but that many universal design enthusiasts requests a thorough understanding and inclusion of accessibility requirements early in the planning phase of the project.

These reflections can also be made in regards to ICT-solutions which often focus on Universal Design late in projects, if it is a focus at all. 

\subsection{Meetups with "Universal Design and digital inclusion"}
To get a glimpse into how the culture is amongst accessibility enthusiast, I went to two meetups organised by the Facebook group “Universal Design - best practises” which I was invited to by a informant I interviewed earlier in the project. 

The first meetup was called 

%\section{Litterature review}
%I have reviewed litterature from several sources. Mainly, I have found articles on ACM and Sage JournalSage Journal




%[TODO: implement theory into the description of the focus group]




\subsubsection{Interview with Universal Design Enthusiast}
% \subsection{Location}
An expert interview was held at a café November 16. 2017 in Oslo city centre. The expert interview was held in relation to the course "INF5261 - Development of mobile information systems" by me and my two group members project in that course. The project had the same theme as this thesis, and the information uncovered in this interview is therefore relevant to this thesis as well. We were three researchers and one participant present during the interview.

The interview was based on an interview guide we had prepared in advance. The interview guide was designed with inspiration from \textcite{tone_nordbo_introduksjon_2017}.

\subsection{Ethics}
A consent form was used to inherit the participant's anonymity and to inform him about the purpose of the interview and the research project. The consent consisted of information about the purpose of the interview, what happens to the information uncovered in the interview, and our contact information - in case he had any questions at a later time. It also informed the participant that he could at any time withdraw from the study without any reason or repercussions. The subject was sent the consent form the day before the interview, so he could read the document in peace and not feel rushed to sign something he was not fully aware of.

\subsection{Participant}
Our expert currently worked as a consultant at a renowned Norwegian firm as a senior front-end developer consultant and manager. He has previous experience in leading a Universal Design disciplinary group at his firm, and considers himself a Universal Design enthusiast. The expert was familiar with MediaLT from collaborating with them on earlier projects.

\subsection{Recording}
The interview was not recorded by audio or video equipment. One of the researchers acted as an observer and note-taker and did not participate in the interview. 

\subsection{Questions}
This is a brief overview of the themes discussed in the first expert interview. For more details, see xx Interview Guide or %\ref{ap-2-expert-interview-1}. 
The interview guide consisted of the following themes:
\begin{itemize}
    \item Introduction and background
    
    Presenting the informed consent form. Present the researchers, time frame, note taking and basic information about the participant.
    
    \item Education
    Asking how the expert got into UU and when they were made conscious about UU.
    
    \item Practising
    
    How the developers in the experts company learns about UU.
    \begin{itemize}
        \item How do you work with Universal Design in your company?
        \begin{itemize}
            \item Insight
            \item Prototyping
            \item Evaluation
        \end{itemize}
    \end{itemize}
    \item Whip or carrot
    
    Asking if laws and repercussions are the best way to make more people focus on UU.
    
    \item Possible solutions
    
    Discussing possible solutions.
\end{itemize}

\subsection{Focus Group with developers}
I participated at a focus group as an observer 03. October 2017. The goal of the focus group was to explore if and how developers relate to Universal Design. This focus group was held by a group doing a project with MediaLT in the course "INF5261 - Development of mobile information systems and services" at the University of Oslo, which I also took at the time.

This focus group was not held by me, I had no impact in the recruitment of the participants, discussion, questions asked or themes brought up. I was mostly an observer. However, I was asked if I knew of any female front end developers that they could recruit to the focus group.

\subsection{Participants and recruiting}
There were 7 participants in the workshop. All of the participants were males between 20 and 27. Most of the participants was working as front-end developers, one were studying programming, and one was looking for a job as a back-end developer. One participant was working as Staffing Manager, and had to consider Universal Design when he accepted projects. All the participants had or was undergoing a bachelors degree in information technology.

The participants were contacts of the students who held the focus group. Most were previous classmates of two of the students who held the focus group.

All the participants were male, but this was not a conscious decision by the focus group-holders, they wanted to recruit more females to balance the discussion.

\subsection{Moderation}
In the focus group, one of the students acted as a moderator. She introduced the theme of the focus group and guided the discussion.

\subsection{Group dynamics}
In the focus group, some people were more audible than others. The moderator did a good job in letting everyone participate in the discussion, and was also actively including all the members by asking them what they thought about the themes discussed.



\subsection{Interview with researcher}
The researcher I interviewed has been working as a backend developer before he decided to become a researcher. His interest field is, amongst other, Universal Design.

\subsection{Interview with developer}
I contacted a person 