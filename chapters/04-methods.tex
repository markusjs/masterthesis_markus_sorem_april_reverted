\chapter{Methods}
In this chapter I present the methods I have used to gather and analyse empirical data. I have only used qualitative data collection methods. This thesis has been exploratory, meaning that the goal was not to end up with a solution, but to explore the field of universal design of ICT solutions.

\section{Research paradigm}
This thesis belongs in the critical research paradigm. According to \textcite{myers_set_2011} critical research in information systems is "...concerned with social issues such as freedom, power, social control". 

As the theme of my research question is to question why so few ICT solutions in Norway are meeting the requirements for Universal Design, and that the goal of Universal Design of ICT is to provide equal access to ICT services for most individuals of a society, the critical paradigm is a good fit for this project. 

Critical research aims to help realise human potential by "...eliminating causes of unwarranted alienation and domination" \textcite{myers_set_2011}. In this context, elimination of causes of unwarranted alienation and domination would be to figure out how to software simulation can help the target users in making ICT solutions which follows Universal Design guidelines. 

\section{Activities}
To understand the universe of universal design of ICT solutions, I have talked to domain experts, participated at universal design relevant gatherings and talked to some developers. Following is an overview of activities I have conducted and an explanation of each activity:
\begin{table}[ht]
    \rowcolors{2}{gray!30}{white}
    \centering
    \caption{My activities}
    \label{my-label}
    \begin{tabular}{ p{5.5cm} p{5cm} p{2cm} }
    \hline
    \rowcolor{gray!50!} \textbf{Participant(s) / setting} & \textbf{Method} & \textbf{Section}\\
    Recent graduated developers & Focus group & \ref{focusgroupdevs} \\
    UD Enthusiast & Interview & \ref{interviewenthusiast} \\
    Meetup 1 & Observation & 4.5 \\
    UD Experience Conference & Observation & 4.5 \\
    Expert & Interview & 4.5 \\
    1 developer & \begin{tabular}[c]{@{}l@{}}Interview \\ Testing simulators\end{tabular} & 4.5 \\
    Meetup 2 & Observation & 4.5 \\
    3 developers & \begin{tabular}[c]{@{}l@{}}Interview \\ Testing simulators\end{tabular} & 4.5 \\
    \end{tabular}
\end{table}

\subsection{Focus group with recent graduated developers} \label{focusgroupdevs}
A focus group was held by fellow students of mine in relation to a course at the University of Oslo. I did not hold the focus group myself, but participated as an observer.

The aim of the focus group was to look at the views and knowledge inexperienced developers had on universal design, and how an ICT solution might support them in learning more about the subject. I wanted to get a better understanding of the views developers has on the subject of universal design and to get ideas on where I might steer the thesis.

Seven participants participated in the workshop. Most of the participants were working as front-end developers, one was studying programming, and one was looking for a job as a backend-developer. All of the participants who were developers were at junior level, meaning they did not have many years of professional experience. One participant was working as Staffing Manager, and had to consider Universal Design when he accepted projects. All the participants had a bachelor degree in informatics, except one that was still studying informatics.	

The participants were contacts of the students who held the focus group. Some were previous classmates of two of the focus group-holders.

One person took notes and no recording was made. In the notes, the participants' quotes were not separated, so it is impossible to state which of the participants said what. 

\subsection{Interview with UD Enthusiast} \label{interviewenthusiast}
An expert interview was held at a café in Oslo city centre. The expert interview was held in relation to the course "INF5261 - Development of mobile information systems" by me and my two group members project in that course. The project had the same theme as this thesis (programmers relation to universal design of ICT), and the information uncovered in this interview is therefore relevant to this thesis as well. We were three researchers and one participant present during the interview.
\subsubsection{Interview guide}
The interview was based on an interview guide we had prepared in advance. The interview guide was designed with inspiration from \textcite{tone_nordbo_introduksjon_2017}. The interview guide consisted of:

\iffalse
\begin{table}[ht]
\begin{center}
\begin{tabular}{|c|c|c|}
\hline
Title 1 &amp; Title 2 &amp; Title 3\\
\hline
1.234 &amp; 5.687 &amp; 2.234\footnotemark[1]\\
1.234 &amp; 5.687\footnotemark[2] &amp; 2.234\\
1.234 &amp; 5.687 &amp; 2.234\\
\hline
\end{tabular}
\end{center}
\end{table}
\footnotetext[1]{First footnote at the bottom of the page.}
\footnotetext[2]{Second footnote at the bottom of the page.}
\fi

%\iffalse
\begin{savenotes}
    \begin{table}[ht]
        \rowcolors{2}{gray!30}{white}
        \centering
        \caption{Interview Guide UD Enthusiast}
        \label{ud-enthusiast-guide}
        \begin{minipage*}{\linewidth}
        \begin{tabular}{ p{5.5cm} p{8cm}}
        \hline
        \rowcolor{gray!50!} \textbf{Part} & \textbf{Themes discussed}\\
        Introduction & Presenting us and the aim of the interview. Signing consent form.  \\
        Main Part & Learning, Practising, Whip or carrot, Impairment simulation \\
        Solutions & Discussing possible solutions \footnote{The course this interview was related to }.  \\
        %\begin{threeparttable}[b]
        %\end{threeparttable}
        Rounding off & Asking if the participant had any additional comments or questions \\
        \end{tabular}
        \end{minipage*}
    \end{table}
\end{savenotes}
%\fi


%Hallo \footnotetext{Since this interview was a part of the university course "INF5261 - Development of mobile information systems and services" }

\subsection{Meetup 1 - How can developers and testers contribute more to accessibility evaluations?}
To meet universal design enthusiast and to get a glimpse on how the culture is within this community, I went to a meetup organised by the Facebook group "Universal Design - best practises". I was invited to this group by the UD enthusiast mentioned in \ref{interviewenthusiast}

The topic for this meetup was for Norwegian Computing Center, NR to present results from a science project were conducting trying to figure out what developers and testers (people concerned with accessibility evaluation of websites and apps) thinks about different ways of doing accessibility evaluations, and which evaluation method is more cost / beneficial and can uncover most accessibility problems.

The statement they made as motivation to conduct the research: 
\begin{itemize}
    \item Few developers and testers has competence in Universal Design, and carries out user testing or accessibility testing in software development.
    
    \item Familiarizing with standards and tools needed for accessibility evaluation can be time consuming.
    
    \item Navigating excisting litterature on UD can be time consuming,
\end{itemize}
\begin{displayquote}
    Few developers and testers has competence in Universal Design, and carries out user testing or accessibility testing in software development.

    It is time consuming to familiarise themselves in standards and tools needed and to navigate existing literature. This mostly concerns developers who is experts at software development, but also testers with more focus on usability and not universal design or accessibility.
\end{displayquote}