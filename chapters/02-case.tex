\chapter{Case}
This thesis has been related to two different projects, MediaLT's "Stable Test Sites" and with an ongoing project in the University of Oslo.
\section{MediaLT}
MediaLT is a Norwegian company concerned with making the web accessible for most people, including people with different kinds of disabilities. They provide training in (among other things) Universal Design and they analyse and comes with suggestions on websites regarding Universal Design principles and implementation.

\subsection{Stable Test Sites}
MediaLT's project "Stable test sites" came together after Difi in 2016 wanted an overview of how reliable automated testing tools are for measuring WCAG 2.0 success criterion. They found out that the only way to measure this was to find inaccessible websites and see if the automated test tools would react on the errors. They soon found out that this was a tedious job, and that it would be much smarter to make their own inaccessible websites that could measure the WCAG 2.0 success criterion.
\section{UDFeed}
UDFeed is concerned with both learning and facilitation of universal design in higher education. The main goal is to focus on feedback from students. Formal evalutation are commonly used to improve upon higher education courses, but this project is concerned with how this feedback can come sooner during the course itself.



\section{Users}
\textit{Users} in this thesis are front-end programmers who has an influence on how an ICT solution aimed at a general population can be designed or implemented. 

%The most relevant users has an ongoing project where they need to cater for universal design in their solution, but as I am also interested in the target users views on disabilities and universal design, all professional designers or front-end developers are target users. 
