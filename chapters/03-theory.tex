%\begin{displayquote}
    %The power of the web is in its universality. Access by everyone regardless of disability is an essential aspect.
    
    %- Tim Berners Lee
%\end{displayquote}

\chapter{Theory}
\epigraphhead[20]{\epigraph{\textit{The power of the web is in its universality. Access by everyone regardless of disability is an essential aspect.}}{{\textbf{Tim Berners-Lee} \\ Inventor of the World Wide Web}}}

\section{Human-Computer Interaction (HCI) and Interaction Design}
\acrshort{hci} is an academic field concerned with "understanding the influence technology has on how people \textit{think, value, feel and relate}" and how this understanding can inform the design of technology \parencite{wright_empathy_2008}. 

%Both HCI and Interaction Design is concerned with understanding people and how they use ICT solutions and how the user experience can be improved in existing solutions or intact/integrated in future solutions.

One of the earliest mentions of the term HCI was in 1976 in a research paper about office automation \parencite{Carlisle1976}. Carlisle problemize the trend of integrating technology into the workspace without acknowledging how people within the organisation do their work and how human beings function and socialise. 
\begin{displayquote}
    Too many computer-based systems have already been designed on the basis of technological breakthroughs and innovations which were insensitive to the limit on man's rationality and the social needs that must be satisfied within organizational structures.
\end{displayquote}

Carlisle thought the implementation of computers in the workplace was backwards. The focus was on hardware and software, not on how people work or function. 

%Interaction Design is the industrial adaptation of HCI research, and is concerned with the practical design of products with the ultimate goal of supporting people (end users) in their everyday and working lives \parencite{rogers_interaction_2011}.
Interaction design is concerned with the practical design of products with the ultimate goal of supporting people (end users) in their everyday and working lives \parencite{rogers_interaction_2011}.

\section{Usability}
Usability is defined in ISO 9241 as 
\begin{displayquote}
    The extent to which a product [service or environment] can be used by specified users to achieve specified goals with effectiveness, efficiency and satisfaction in a specified context of use \parencite{Petrie2013}.
\end{displayquote} 

\textcite{rogers_interaction_2011} has extended the notion of usability with: \textit{safety}, \textit{utility}, \textit{learnability} and \textit{memorability}.

%The degree of usability is measured in terms of \textit{effectiveness}, \textit{efficiency} , \textit{safety}, \textit{utility}, \textit{learnability} and \textit{memorability} \parencite{rogers_interaction_2011}. Effectiveness 

%\section{User Experience}
%User experience is a term used for how an artefact behaves and is used in the real world \parencite[]{rogers_interaction_2011}. \textcite{garrett2010elements} says user experience design is to make sure  aesthetic and functional aspects of an artefact work in context of the rest of the artefact and in context of what the user is trying to accomplish.

%User experience is about how people feel about a product and the pleasure or satisfaction (or lack of it) when they use it. User experience governs more than usability 

%\section{Usability}





%The term "usability" is central to user experience
%ISO 9241 defines usability as \begin{displayquote}
%The extent to which a product [service or environment] can be used by specified users to achieve specified goals with effectiveness, efficiency and satisfaction in a specified context of use.
%\end{displayquote}

\subsection{Accessibility}
ISO 9241-171 defines accessibility as:
\begin{displayquote}
    The usability of a product, service, environment or facility by people with the widest range of capabilities.
\end{displayquote}

In this definition, accessibility can be seen as a subset of usability with the inclusion of people with the widest range of capabilities.


%Web accessibility means that websites, tools, and technologies are designed and developed so that people with disabilities can use them. More specifically, people can:

%perceive, understand, navigate, and interact with the Web
%contribute to the Web
\section{Universal Design}
%Universal design of ICT means that the same solution should be usable by all people 

The Norwegian Environmental Protection Agency defined Universal Design in 2007 as “...the design of products and environments to be usable by all people, to the greatest extent possible, without the need for adaptation or specialised design” \parencite{miljoverndepartementet_t-1468_2007}. This means that the same solution should be usable for people with different functional levels.

\subsection{Teaching Universal Design}



%\textcite{Jordan2010} says that the current practise in the education programs for engineers and ICT students have been is lacking information about universal design and people with disabilities. As a result, few professionals will have been exposed to the barriers faced by people with disabilities and likely maintain them. 

%Disability awareness has the aim of changing attitudes towards people with disabilities. Negative attitudes can be expressed "through avoidance, anxiety, overprotectiveness, pity, segregation, alienation, and rejection" (Elliot \& Byrd 1982 in \cite{Jordan2010}). Jordan request the need for disability awareness in the engineering and ICT programs.

\textcite{Jordan2010} lists four broad approaches to learn about and change attitudes towards people with disabilities: education, facilitated contact with people with disabilities, role playing and disability simulation: 
\begin{itemize}
    \item Education for people without disabilities
    
    Accurate information is used to inform and create understanding on disabilities.
    \item Facilitated contact with people with disabilities
    
    Direct contact with people who experience disabilities. This can challange the students preconceived thoughts on disabilities. 
    \item Role Playing
    
    Students are put to imagine and take the role as a disabled person in order to empathise with some of the experiences they might have.
    \item Disability simulation
    
    Physical or sensorial limitations are used while activities are conducted.
\end{itemize}

\textcite{Jordan2010} argue that education, facilitated contact and role playing require a significant amount of time, but that disability simulations can be quick and effective to create awareness.
%\section{Empathy tools}
%While there have been many mentions of simulation tools in the litterature \parencite{GoodmanDeane:2007it, Giakoumis2014, Cardoso2012}
%While there have been many mentions of simulation tools in the litterature, I have only 
\section{Simulations}
According to \textcite{Cardoso2012}, in the field of inclusive / Universal Design \textit{simulations} refer to the use of physical restrainers that enables a person to feel the effects different types of capability-losses might have on a person. The aim is to alter the wearers’ experience of their environment to show how everyday products often disregard (and hence disable) a large number of users, due to a lack of consideration of their capacities.

%Software simulations where the use of filters and modification of 

One of the earliest documented use of a simulator in design was in the 1950s. A group of industrial designers wore artifical limbs to empathise with war veterans who had amputated their limbs \parencite[3]{Cardoso2012}. In the 1970s, an approach to design called \textit{empathetic modellling} was used in reaserch and design. Empathetic modelling aims to "take designers out of their comfort zone". 
\begin{displayquote}
    Empathic modelling offers designers the opportunity to develop greater insight and understanding, in order to support more effective design outcomes.
    
    We all approach others with our own assumptions and preconceptions until we learn something different or contradictory. This is why empathic modelling is so important; it pushes people to better understand their own values and belief systems, which result in a move towards less ‘projection’ of their own perceptions onto others.
\end{displayquote}
\section{What can be simulated?}
Visual impairments can be simulated using glasses such as 

\subsection{Critique on using simulations}
\textcite{French1992} is very critical to the use of impairment simulators to bring positive attitudinal changes towards disabled people. She says that such simulations seems like a good idea, but that they are often harmful and counter-intuitive. She points to one study where student nurses spent a day in a wheelchair and felt a lack of self-esteem and feeling sexually unattractive. 

\textcite{French1992} says that by depriving able-bodied people of one of their senses will definitely provide difficulties and fear, and that able-bodied people know that they can "go back" to their able-bodied status. 

Simulation excercises can also provide false impressions that disabled people are heroic and superhuman for managing their lives living with a disability. This can lead to very damaging and direct the focus on the disabled person rather than the society and hostile environment they face. 

\textcite{young_im_2014} also reports on this view some people have. She has been given awards just for being "brave" as she is in a wheelchair. She says that this is objectifying disabled people, as non-disabled people tend to think "Well, however bad my life is, it could be worse. I could be that person.". She calls this "inspirational porn".

%\textcite{Riccobono2018} addresses the issue of reading too much into a simulation excersise. 
\textcite{Riccobono2018} says that simulations of impairments can be effective, but only if it is used the right way. If used incorrectly, simulation excercises can do more harm than good. He illustrates his point by an example: 
\begin{displayquote}
    Ask a newly blindfolded person to travel the streets so she comes to understand the value of traffic sounds, and her predominant emotion will be fear.
\end{displayquote}

This fear comes out of the newly blindfolded person temporary experience of loosing one of his main senses, without knowing how a blind person crosses the street. But if that newly blindfolded person has observed how a blind person crosses a streets using a cane, and seeing how \textbf{the environment} can be changed to his particular way of manoeuvring and understanding how he uses his cane (what it can and cannot detect) it becomes clear what alterations to the environment can be done in order for him to cross the street.

\subsubsection{Humourising disability}
\textcite{French1992} claims that disability simulations can appear funny to some people, as it is the response some people might have during simulation exercises. \begin{displayquote}
    People undergoing simulation excercises do not appear to perceive themselves as disabled, but rather as the participants of a funny game, like "blind man's buff.
\end{displayquote} 

This response can be very offensive to disabled people and taken as mockery.

\subsection{Self determination theory (SDT)}
Self Determination Theory, developed by Edward Deci and Ryan Richard, aims to explain which factors tend to elicit and sustain the human natural need for being curious, to explore and to extend and exercise one's capabilities (Deci 2000). In clear terms SDT aims to explain what factors need to be present for a person to be motivated. Essential for this theory is that human beings have a natural need for \textit{competence}, relatedness and autonomy.

Competence refers to feeling confident and affective in relation to the task you are doing. Relatedness refers to feeling cared for by others and to care for others. Autonomy is the feeling of being control over you own actions.

SDT uses two terms for motivation: intrinsic and extrinsic. Intrinsic motivation is the inner drive, doing something because you enjoy the activity itself. Extrinsic motivation is the drive to do something that leads to a seperate consequense. SDT 


\section{Related work}
A lot of different studies has been conducted where simulations in some form has been viewed as a way to increase understanding and empathize with people different than the person using the simulator. 

\subsection{Simulation in user-centred design: Helping designers to empathise with atypical users}
\textcite{Cardoso2012} says that designers often applies self-observation teqhniques when they design. This can lead to difficult, frustrating, dysfunctional or even dangerous interactions for a wider variety of people. People experiencing capability-losses, including old and people with temporary or permanent impairments, are likely to be most affected. A user-centered approach to design advocates for direct contact with end-users throughout the design process. 

However, sometimes clients might not think user involvement is a priority, and no allocation is made for it in the budget. Assessing with the use of user simulations has been used in times where user involvement can be hard to accomplish: 
\begin{displayquote}
    In the field of Inclusive/Universal Design, this has consisted of a person (usually able-bodied) wearing physical restrainers to feel the effects of different types of capability-losses, experienced for instance by people with impairments.
\end{displayquote}

%Different impairments can be simulated:

%The aim is to alter the wearers’ experience of their environment to show how everyday products often disregard (and hence disable) a large number of users, due to a lack of consideration of their capacities.

\subsection{Simulating REAL LIVES: Promoting Global Empathy and Interest in Learning Through Simulation Games}
The aim of this study was to investigate if a game that simulates the life of people from different parts of the world could impact students empathy for people of other nation-states, etnisity and linguisticly.

The researchers has found some evidence of positive outcomes of simulations and empathy excercises:
\begin{itemize}
    \item Quin, Rau and Salvendy found that empathy is a dimension of immersion in game narratives. This can mean that games that offers the player to empathise with the characters are more engaging.
    \item Shieh and Cheng found evidence for empathy contributing to greater satisfaction in games.
    \item Yee and Bailenson conducted research on the use of virtual reality simulation and perspective-taking. Young people embodied the character of either an elderly person or a young character. The researchers concluded that those who embodied the elderly character had fewer stereotypical views on elderly and made more situational over dispositional remarks on the persons actions. 
    
    This might indicate that simulations can impact a shift from the medical to the social model of viewing disabilities [my own remarks].
\end{itemize}

REAL LIVES is a game that simulates how it would be to live as a person in another country. The game is text-based and uses real-world data. The game is aimed at 15 year old teenagers and shows how a life can be expected to unfold. 

The hypothesis of the study:
\begin{enumerate}
    \item Students who play the simulation game will exhibit greater global empathy compared with those in an alternate computer-assisted learning activity
    \begin{itemize}
        \item Supported
    \end{itemize}
    \item Students who play the simulation game will show greater interest in future learning about the countries studied compared with students in the alternate computer-assisted learning activity
    \begin{itemize}
        \item Supported
    \end{itemize}
    \item Character identification will be positively associated with global empathy for those who play the simulation game.
    \begin{itemize}
        \item Supported
    \end{itemize}
\end{enumerate}

The results showed that students who played REAL LIVES showed significantly higher levels of global empathy. The students interest for learning more about the lives of people in other nations were also significantly higher. Students showed interest three weeks after being exposed to the simulation excercise. The students were also observed cheering when their character did well and shouting in dismay when negative events occured which might mean that they identified with their character.

