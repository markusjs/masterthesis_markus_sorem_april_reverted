\chapter{Results}

\section{Learning about Universal Design}
\subsection{Prior assumptions and views on universal design in ICT}
Many of the participants of the initial focus group had heard of Universal Design, but few knew exactly what it means or how to implement it. 

Some of the focus group participants regarded responsive design (making websites scale up or down to different device sizes) as the main universal design adjustments they do, and some mentioned alternative texts. They thought most of the universal design responsibility was up to interaction designers.

\subsection{Universal Design in higher ICT education}
Some participants of the initial focus group (see \ref{focusgroupdevs}) requested a better focus on Universal Design in their own education, as it would make their jobs easier if they had had more about the topic during their education.

When asked about if and when they learned about Universal Design during their studies:
\iffalse
\begin{displayquote}
    "I had about it in one course in second year. Most of it has been forgotten."
    
    "The first year we had a web design course where the last delivery had a requirement to be Universal Designed. Then I had to read up on WCAG requirements."
    
    "I had about it in one course in second year. Most of it has been forgotten."
    
    "If you should have it in school, you should learn it early on. You should learn it early, not in the last part of the education. If you do that, you get the same situation we have today where you design something first, and then add Universal Design at the end."
\end{displayquote}
\fi
%\iffalse
\begin{table}[!ht]
    \rowcolors{2}{gray!30}{white}
    \centering
    \caption{UD in education}
    \begin{tabular}{p{12cm}}
        \rowcolor{gray!0!} \textbf{Statement:} \\
        "I had about it in one course in second year. Most of it has been forgotten." \\
        "The first year we had a web design course where the last delivery had a requirement to be Universal Designed. Then I had to read up on WCAG requirements." \\
        "No, we didn't have that in the curriculum, there's no courses for that." \\
        %"If you should have it in school, you should learn it early on. You should learn it early, not in the last part of the education. If you do that, you get the same situation we have today where you design something first, and then add Universal Design at the end."
    \end{tabular}
    
    \label{tab:my_label}
\end{table}
And when asked if they think unviersal design should be a bigger part of higher education of ICT students:
\begin{table}[ht]
    \rowcolors{2}{gray!30}{white}
    \centering
    \caption{Caption}
    \begin{tabular}{p{12cm}}
        \rowcolor{gray!0!} \textbf{Statement:} \\
        "It is a theoretical important piece that could have been nice to learn about in school" \\
        
        "Yes, it can be nice to have in mind from the start, so it isn't something you do the last week before you hand in a project." \\
        
        "If you should have it in school, you should learn it early on. You should learn it early, not in the last part of the education. If you do that, you get the same situation we have today where you design something first, and then add Universal Design at the end."
    \end{tabular}
    
    \label{tab:my_label}
\end{table}
%\fi

%most of the participants answered that they had little to no courses that focused on Universal Design. Only a minority had courses where UD were a topic, and most of them were in courses that taught web development. Most of the developers regarded responsivity (making interfaces responsive to different screens) as their main UD-task when doing development, and thought the remainder of the UD-work was the job of the Interaction Designer.


\iffalse
\subsection{Focus group with recent graduated developers} \label{focusgroupdevsres}
Many of the participants had heard of Universal Design, but few knew exactly what it means or how to implement it. The developers did as they were told by their project leaders, and if they were requested to implement an accessibility fix, they did that. Some participants requested a better focus on Universal Design in higher education, as it would make their jobs easier if they had had more about the topic during their education.

When asked about if and when they learned about Universal Design during their studies, most of the participants answered that they had little to no courses that focused on Universal Design. Only a minority had courses where UD were a topic, and most of them were in courses that taught web development. Most of the developers regarded responsivity (making interfaces responsive to different screens) as their main UD-task when doing development, and thought the remainder of the UD-work was the job of the Interaction Designer.
	
Initially, there was little motivation with regards to learning more about Universal Design among the developers, unless they were forced to. Ideally they would have a type of automated developer tool that did the work for them. However, we found comfort in a statement of one of them towards the end of the workshop: “[...] it can be nice to have in mind from the start, so that it is not something you do the last week before handing in the project”. Some of the others agreed, and added that it should be taught in some of the earlier courses, in order to think about it when building the foundation of an application, rather than adding it on top. 

%When asked about the possibility of learning Universal Design through a quiz, some of the participants liked the idea, while others thought that this would be information they would forget about after a short period of time. One of the participants stated that: “I might learn about Universal Design now, but I may not need the information until half a year from now. It would be nice with a type of encyclopedia”. 
\fi

\subsection{Easy to learn, hard to maintain}
%The workshop participants said they would use Funkify in their project if they



