\chapter{Discussion}
Sven cannot expect the Norwegians he meets to speak Swedish to him, and Norwegians can't expect Sven to understand all the dialect words they might normally use in conversations with other Norwegians. This mutual understanding comes from interacting with each other.

At the same time, developers can't expect every user to have the same functional capabilities as themselves. Using empathy tools they can get a sense of which barriers the solution might have. If the alternative is that the developer has no contact with real users, empathy tools is definitely a good alternative. The best alternative is for the developer to see how people with different capabilities other then themselves are using the ICT solution.


\textbf{In what way can impairment simulators be used in ICT projects?}

To answer this, I would like to clarify that empathy tools can be used in different parts of an ICT project \parencite{Keates2014}:
\begin{itemize}
    %\item Gain empathy to
    \item Formative evaluations
    \item Summative evaluation
\end{itemize}
During the concept development phase, simulations can be used on similar solutions to see what can be improved up. Inspiration can also be taken from similar solutions as another solution might have found a way to accommodate for example for people who have a hard time reading text. Simulations can make it more clear what works in another solution.
    
Simulations can be used on prototypes to make sure the solution is not making any apparent accessibility barriers for some users \parencite{Keates2014}.

Simulations can also be used to evaluate a solution when it has been made. However, it must be used with an understanding of how people with different impairments use the solution and why the simulations are used.  

%\subsection{\textit{How to convey the message of universal design as something bigger than technical accessibility?}}

\section{Relatable examples}
Stian's use of relatable examples in his teaching can give the developers some perspective before they are exposed to the legal requirements. 

%Instead of focusing on requirements and how to make ICT solutions that are legal and functional accessible, he uses examples where everyone could benefit of more accessible ICT solutions.




%\section{\textit{Can empathy tools motivate developers to consider the needs of people with different capabilities?}}




\section{\textit{How does the use of empathy tools impact developers?}}

%\section{\textit{Which kinds of empathy tools are best suited for developers working on ICT
%solutions?}}

\subsection{\textit{Which kinds of empathy tools are best suited for developers working on ICT
solutions?}}
For simulators to be a tool for developers, it needs to be accurate and integrated in the current environments of the developers.

\subsection{Cost effective}
Cutting costs and saving time is an important aspect of ICT projects.

\begin{displayquote}
    People are always concerned with cutting costs. There are no-one that, unless they need to or make money of it, do something. That's how the reality is. We want to do the best job, in a pragmatic way. But the customer has their budgets to relate to that they might not be in charge of themselves.
\end{displayquote}
With this in mind, for a simulator to be useful for developers it would need to be calibrated with statistics so that developers are sure to include the widest range of possible end-users, and that the solution is within the legal requirements. 

\begin{displayquote}
    I know there is a list, and that not all the requirements need to be fullfilled. So it would be cool if you could turn on WCAG requirements based on where you are, Norway, America and so on. 
    
    I think that is very relevant, especially for professional use. This [Funkify and Cambridge glasses] is cool as a thing, but if you are selling it to professional use, you have to be able to configure this after the teams needs.
\end{displayquote}